%-------------------------------------------------------------------------------
% IPOL LaTeX class manual
% by rafael grompone von gioi
% ver 0.5 - July 1, 2014
%-------------------------------------------------------------------------------
\documentclass{ipol}

\ipolSetTitle{IPOL \LaTeX\ Class Manual, ver.0.5}
\ipolSetAuthors{Rafael Grompone von Gioi\ipolAuthorMark{1},
                Rapha\"el Toucour\ipolAuthorMark{2}}
\ipolSetAffiliations{%
\ipolAuthorMark{1} CMLA, ENS Cachan, France
                   (\texttt{grompone@cmla.ens-cachan.fr})\\
\ipolAuthorMark{2} IIE, UdelaR, Uruguay (\texttt{jirafa@fing.edu.uy})}

%-------------------------------------------------------------------------------
\begin{document}

%-------------------------------------------------------------------------------
\begin{ipolAbstract}
This document describes and illustrates how to use the IPOL \LaTeX\ class,
created to produce a uniform layout for IPOL articles. The restrictions imposed
on articles to be published in IPOL are briefly discussed.
\end{ipolAbstract}

%-------------------------------------------------------------------------------
\ipolKeywords{IPOL, LaTeX class, documentation}

%-------------------------------------------------------------------------------
\section{Getting Started}

There is no need for installation to use this class, just two files need to be
copied to the same directory were the \LaTeX\ source files are. These files are
the class itself, \verb|ipol.cls|, and the logo file, that can be either
\verb|ipol_logo.eps| if you compile with \verb|latex|, or \verb|ipol_logo.pdf|
if you compile with \verb|pdflatex|. If not sure, just copy the three of these
files. You may also need \verb|siims_logo.eps| or \verb|siims_logo.jpg| when
preparing a SIIMS companion article.

The minimal example of use of this class is as follows:
\begin{verbatim}
\documentclass{ipol}
\begin{document}
\end{document}
\end{verbatim}
It will only generate the IPOL header, including its logo, the words ``title''
and ``authors'' where the title and authors should be placed, and a red
``PREPRINT'' label including the compilation date. This example is useless but
can be used to test your system. The \LaTeX\ source file
\verb|ipol_class_manual.tex|, which generates this manual, also uses the IPOL
class and provides an example of how to use it.

IPOL class is based on the standard ``article'' class of \LaTeX\ and it is used
essentially in the same way. There are two main restrictions: the layout must
not be changed and the title is not generated with the usual \verb|\title|,
\verb|\author|, \verb|\date| and \verb|\maketitle| commands; special IPOL
commands must be used instead. Let us discuss these topics one by one.

This class was created to provide a uniform layout for IPOL articles, so no
command should be used that would change the page layout: do not change paper
size (it must be A4 paper), do not change the page margins, do not change the
font type, size, or color.

The commands \verb|\ipolSetTitle|, \verb|\ipolSetAuthors| and
\verb|\ipolSetAffiliations| are provided by the IPOL class to set the
information needed to generate the title of the article. These commands must be
used in the preamble of the \LaTeX\ file, that is between the
\verb|\documentclass{ipol}| and \verb|\begin{document}| commands; this is
important, otherwise the title will not be generated correctly. As the name of
the commands imply, \verb|\ipolSetTitle| sets the title of the article,
\verb|\ipolSetAuthors| sets the authors, and \verb|\ipolSetAffiliations| sets
the authors' affiliations. This last command is optional. The following is an
example of how to use it:
\begin{verbatim}
\documentclass{ipol}
\ipolSetTitle{Ceci n'est pas un article}
\ipolSetAuthors{Rapha\"el Toucour}
\ipolSetAffiliations{CMLA, ENS Cachan, France}
\begin{document}
\end{document}
\end{verbatim}
If there are more than one author, and they have different affiliations, they
may be indicated using the command \verb|\ipolAuthorMark| (authors must be
separated by commas, never by ``and'' or \&):
\begin{verbatim}
\documentclass{ipol}
\ipolSetTitle{Ceci n'est pas un article}
\ipolSetAuthors{Rapha\"el Toucour\ipolAuthorMark{1},
                Rafael Grompone von Gioi\ipolAuthorMark{2}}
\ipolSetAffiliations{\ipolAuthorMark{1} CMLA, ENS Cachan, France\\
                     \ipolAuthorMark{2} IIE, UdelaR, Uruguay}
\begin{document}
\end{document}
\end{verbatim}
For the rest, the class is used as the \LaTeX\ standard ``article'' class:
packages or definitions can be added in the preamble, and the main text can
contain sections, subsections, mathematics, figures, tables, etc.

The rest of this document describes some special command for IPOL and briefly
comments on some conditions that will be imposed for a final article, just
before publishing at IPOL. (These restrictions are not necessary for
submissions but will reduce work later.)

%-------------------------------------------------------------------------------
\section{IPOL Commands for Authors}

%-------------------------------------------------------------------------------
\subsection{DOI and Links}

Once an article is published in IPOL, it will be assigned a DOI number and a
universal DOI web address to refer to it. The IPOL class provides a command
that will later contain the article's DOI number (\verb|\ipolDOI|) and a
command that will contain the article's web address (\verb|\ipolLink|). These
are the only reference that should be used for the IPOL article itself and to
its complementary content on the web.

These commands will have a default value in a preprint version, but once the
paper is accepted they will be replaced by the correct one. One may want to
have a valid link while the article is being prepared and peer reviewed. For
that purpose, the command \verb|\ipolPreprintLink| is used to set the article
link \verb|\ipolLink| to a temporary value. For example,
\begin{verbatim}
\ipolPreprintLink{https://tools.ipol.im/wiki/ref/manuscript_guidelines/}
\end{verbatim}
This command should be used in the document's preamble (i.e., after
\verb|\documentclass{ipol}| and before \verb|\begin{document}|).

The command \verb|\ipolLink| provides the address of the corresponding IPOL
article as \emph{text}. To produce a \emph{link} in the PDF with this or other
address, the command \verb|\href{}{}| should be used. The first parameter is
the link address and the second is the text that will be shown in the resulting
document. Also, to be able to read the address in a printed version of the
document, the address will be printed as footnote. For example, the commands
\begin{verbatim}
\href{\ipolLink}{the article page}
\href{https://tools.ipol.im/wiki/ref/manuscript_guidelines/}{guidelines}
\end{verbatim}
produce the following links:
\href{\ipolLink}{the article page} and
\href{https://tools.ipol.im/wiki/ref/manuscript_guidelines/}{guidelines}.

%-------------------------------------------------------------------------------
\subsection{Abstract, Code, Supplementary Material and Keywords}

Three environments are defined to generate specific sections of IPOL articles:
the abstract, the Source Code description paragraph, and the Supplementary
Material description paragraph. The corresponding environments are
\verb|ipolAbstract|, \verb|ipolCode| and \verb|ipolSupp|. These environment
work similarly as the \LaTeX\ ``abstract'' and should be used after the
\verb|\begin{document}| and before the main text of the article. It is
important to use the \verb|ipolAbstract| for the abstract and not the
\LaTeX\ ``abstract''.

There is a particular restriction for the abstract: it must contain only
\emph{one} paragraph. This restriction is necessary to produce the final
article as the abstract is included in the PDF metadata, and this would
generate an error if the abstract contains more than one paragraph.

The command \verb|\ipolKeywords| is used to set the keywords of the article.
The command is followed by a list of keywords separated by commas, as in
\verb|\ipolKeywords{Earth, Water, Air, Fire}|. This command is to be used after
the abstract, the source code paragraph and the supplementary material, but
before the beginning of the article itself.

The following example illustrates the use of the four commands:
\begin{verbatim}
\documentclass{ipol}
\ipolSetTitle{Ceci n'est pas un article}
\ipolSetAuthors{Rapha\"el Toucour}
\begin{document}

\begin{ipolAbstract}
A short description of the article.
\end{ipolAbstract}

\begin{ipolCode}
Description of the source code to be found \href{\ipolLink}{here}.
\end{ipolCode}

\begin{ipolSupp}
Some articles could provide additional material, not part of the
peer reviewed article but related and useful. It should be found at
\href{\ipolLink}{the web page}.
\end{ipolSupp}

\ipolKeywords{IPOL, LaTeX class, documentation}

\section{Introduction}
The article main text starts here.

\end{document}
\end{verbatim}

%-------------------------------------------------------------------------------
\subsection{SIIMS Companion Article}

IPOL encourages authors to do joint submissions to IPOL and
\href{http://www.siam.org/journals/siims.php}{SIIMS (SIAM Journal of Imaging
  Science)}. Upon acceptance, cross links between both electronic articles will
be placed, so readers will be able to automatically navigate between the SIIMS
and IPOL complementary materials. The environment \verb|ipolSIIMS| is used to
set the link. It should be used after the \verb|\begin{document}| but before
the abstract. The following source code provides an example:
\begin{verbatim}
\documentclass{ipol}
\begin{document}

\begin{ipolSIIMS}
This IPOL article is related to a companion publication in the SIAM
Journal on Imaging Sciences:\\
R. Toucour, ``Ceci n'est pas un article.''
\textsl{SIAM Journal on Imaging Sciences}, vol.~X, no.~X,
pp.~N--M, YYYY. \url{http://dx.doi.org/10.1137/nnnnnnnnn}
\end{ipolSIIMS}

\end{document}
\end{verbatim}

%-------------------------------------------------------------------------------
\section{Manuscripts for IPOL}

Authors must check the
\href{https://tools.ipol.im/wiki/ref/manuscript_guidelines/}{IPOL manuscript
  guidelines} where the requirements and suggestions about the contents, style
and description detail for IPOL articles are described.

The PDF files for IPOL submissions are generated by the authors using any
method that produces a readable PDF file. Once an article is accepted, however,
the final version will be generated at IPOL servers in order to provide PDF
files with controlled sizes and resolutions conditions. For that aim, there are
some restrictions that must be imposed to the source files. Please check
\href{https://tools.ipol.im/wiki/ref/manuscript_guidelines/}{here} for a
complete list of restrictions. The main consideration is that the final
documents will be generated using pdf\LaTeX. This forces some restrictions, in
particular the only graphical formats directly accepted are PNG, JPEG and PDF
itself. Images or figures in other kind of formats should be converted to one
of these. The files and directories must follow a strict naming as defined
\href{https://tools.ipol.im/wiki/ref/manuscript_guidelines/}{here}.

Authors must warrant to be the authors of every part of the final article, or
must include permission of the copyright holder. In particular, if images not
owned by the authors are used, the authors and permissions to be used must be
included in the article. For example, by adding a final section as follows:
\begin{verbatim}
\section*{Image Credits}

\includegraphics[height=2em]{image1.png}
\href{http://a.link.here}{Courtesy of Name Surname}
\\
\includegraphics[height=2em]{image2.png}
\copyright\ Pierre Dupont CC-BY
\\
\includegraphics[height=2em]{image3.png}
the authors
\end{verbatim}

%-------------------------------------------------------------------------------
\section{IPOL Commands for Final Production}

This section describes some commands used for final article production.
These commands \textbf{must never be used by authors}.


%-------------------------------------------------------------------------------
\subsection{Article Final Data}

Before an article is ready for publication, some data need to be declared.
These data are used to create the IPOL header, the article's citation on the
footer of the first page, and are included in the PDF metadata. Part of the
data was already added by the authors as the title, authors and
affiliations. What needs to be added in final production is the submission,
acceptance and publication dates, the article ID, and the first page number.

This is done by including the following commands in the preamble of the
\LaTeX\ file, that is between the \verb|\documentclass{ipol}| and
\verb|\begin{document}| commands. Some random values were including here just
to provide an example:
\begin{verbatim}
\ipolSetSubmissionDay{11}
\ipolSetSubmissionMonth{11}
\ipolSetSubmissionYear{2010}
\ipolSetAcceptedDay{1}
\ipolSetAcceptedMonth{3}
\ipolSetAcceptedYear{2011}
\ipolSetPublicationDay{28}
\ipolSetPublicationMonth{3}
\ipolSetPublicationYear{2014}
\ipolSetID{gjmr-lsd}
\ipolSetVolume{4}
\setcounter{page}{124}
\end{verbatim}
When no data is missing, the ``PREPRINT'' label disappear. If the ``PREPRINT''
label is still present, some data may be missing or the
\verb|\ipolPreprintLink| command was used to set a temporary link for the
article and needs to be removed.

When the PDF is generated in the final article mode (that is when the
``PREPRINT'' label disappear) the article's abstract is included in the PDF
metadata. This could generate a \LaTeX\ error if the abstract is composed of
more than one paragraph. The error generated would be as follows:
\begin{verbatim}
Class ipol Warning: If an error follows, the abstract may have many paragraphs.
 To work correctly, IPOL class requires the abstract to be only ONE paragraph.

Runaway argument?
! Paragraph ended before \def was complete.
<to be read again>
                   \par
l.74 \end{ipolAbstract}

?
\end{verbatim}
A warning was generated by the IPOL class to help the user to spot the source
of the problem. The solution is to merge the abstract into \emph{one}
paragraph.

%-------------------------------------------------------------------------------
\subsection{Corrections to Citation}

With the previous data, the class will generate a citation reference for the
article that is put in the footer of the first page. In some cases, the
automatically generated citation may not be correct or satisfactory. In such
cases, one can force a manual citation using the command
\verb|\ipolForceCitation| as in the following example, to be used in the
preamble of the \LaTeX\ file, that is between the \verb|\documentclass{ipol}|
and \verb|\begin{document}| commands:
\begin{verbatim}
\ipolForceCitation{Rapha\"el Toucour, ``Ceci n'est pas un article,''
\emph{Image Processing On Line}, vol.~2013, pp.~123--124.
http://dx.doi.org/10.5201/ipol}
\end{verbatim}
The standard citation would be obtained with:
\begin{verbatim}
\ipolForceCitation{\ipolAuthors, ``\ipolTitle,''
\emph{Image Processing On Line}, vol.~\ipolPublicationYear,
pp.~\pageref*{ipol:class:first:page}--\pageref*{ipol:class:last:page}.
\ipolLink}
\end{verbatim}
The latter may be useful: it is better to modify just what is wrong and leave
the rest unchanged. For example, unless there is problem there, it is better to
keep the \verb|\pageref*| than manually including the page numbers.

%-------------------------------------------------------------------------------
\subsection{IPOL Class Version}

From IPOL class version 0.3 on, one can easily verify the version of this class
that generated a given PDF: the version is printed in light gray in the
upper-left side of the first page of the document and also included in the PDF
metadata (the latter only if the system that generated the PDF included an
up-to-date version of the package ``hyperref'').

%-------------------------------------------------------------------------------
\section*{Image Credits}

\includegraphics[height=2em]{ipol_logo} \copyright\ IPOL (there's no need to
credit this image, here is used as an example.)

\end{document}
%-------------------------------------------------------------------------------
