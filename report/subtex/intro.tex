%\section{TODOs}
%\begin{itemize}
%    \item For the experiments:
%    \begin{itemize}
%        \item Test the robustness of the algorithm when facing outliers or skewed distributions.
%        \item Use noisy ordinal data (add discrete noise on a few features).
%        \item Test scalability of the algorithm (number of variables and data points).
%    \end{itemize}
%\end{itemize}

\section{Introduction}
The exploration of hidden structures within datasets is a crucial task for data scientists, and clustering serves as a valuable tool in this endeavor. Mixture models have emerged as a standard approach for clustering due to their capacity to provide a well-defined mathematical framework for parameter estimation and model selection. These models, instrumental in determining the number of clusters, not only encapsulate classical geometric methods but also find successful application in diverse practical scenarios.

In the realm of model-based clustering, the classification of data hinges on the availability of a suitable probability distribution tailored to the nature of the data at hand—be it numerical, rankings, functional, or categorical. Notably, ordinal data, where categories possess a specific order, represent a common occurrence, especially in fields like marketing where product evaluations are solicited through ordinal scales. Despite their prevalence, ordinal data have received comparatively less attention in the context of model-based clustering. Often, practitioners resort to transforming ordinal data into quantitative or nominal formats to align with readily applicable distributions, neglecting valuable order information.

This paper explores the less-explored domain of model-based clustering for ordinal data, specifically focusing on ordinal data derived from ordered categories. Ordinal data find widespread application in fields such as social sciences, psychology, marketing, healthcare, and more. They enable researchers to capture nuanced information, such as preferences, attitudes, or severity levels, in cases where continuous measures are neither significant nor possible. For example, when assessing tumor severity, the precise size may not be as crucial as the current state of development of the disease as it is assessed by specialists. The use of ordinal data enriches the comprehension of subjective opinions, behaviors, and hierarchical relationships across diverse research contexts. Over the years, various approaches have been proposed to define probability distributions for ordinal data, including modeling cumulative probabilities, constraining multinomial models to reflect ordinality, assuming ordinal data as discretization of continuous latent variables, and constructing distributions to meet specific properties.

Among these approaches, the work by \cite{biernacki2016model}, studied in the context of this project, delves into an original strategy—modeling the hypothetical data generating process for ordinal data. While this general principle has found success in ranking data scenarios, the distinction in the data generating process becomes apparent for ordinal data. In this context, a search algorithm, specifically the binary search algorithm, emerges as a fitting choice, respecting the ordinal nature of data through comparisons without necessitating links to nominal or continuous distributions.

The proposed model, parameterized with a position parameter (modal category) and a precision parameter, exhibits desirable properties such as a unique mode, probability distribution decrease on either side of the mode, and the flexibility to accommodate uniform or Dirac distributions. Maximum likelihood estimation using an EM algorithm is employed, leveraging the binary search algorithm's latent variable interpretation. While combinatorial complexity limits straightforward estimation for models based on latent Gaussian variables, the proposed approach remains tractable for ordinal data with up to eight categories—a common scenario for most ordinal variables.

In this project, we aim to replicate and build upon the findings of \cite{biernacki2016model}. We re-implemented their suggested probabilistic model, parameter estimation method, and model-based clustering algorithm in Python. Drawing inspiration from their approach, we propose an alternative probabilistic model with similar properties. The goal is to address computational limitations, enabling the clustering of more extensive datasets with potentially more categories than the previous method allows. We also present a preliminary analysis of this new method to justify the decreased computational cost of estimating parameters for this model. Additionally, we  test \cite{biernacki2016model}'s approach on real-world datasets and compare it to the proposed approach, along with baseline models. This is done on different datasets of multiple nature in order to check whether the proposed methods are successful in these settings in practice and what their advantages are. The ultimate goal is to check whether the gains are significant against methods that are not adapted for ordinal datasets, in order to decide whether these approaches are interesting to use in general as a default method of choice for this type of variable.
% \tm {Explain better what we do for the experiments}


