\subsection{BOS Model}

\begin{align}
    \Pr(x | x \in e_j, \mu, \pi) 
    &= \sum_{e_{j+1} \subset e_j} \Pr(x, e_{j+1} | x \in e_j, \mu, \pi) \\
    &= \sum_{e_{j+1} \subset e_j} \Pr(x | e_{j+1}, x \in e_j, \mu, \pi) \Pr(e_{j+1} | e_j, \mu, \pi) \\
    &= \sum_{e_{j+1} \subset e_j ; x\in e_{j+1}} \Pr(x | x \in e_{j+1}, \mu, \pi) \Pr(e_{j+1} | e_j, \mu, \pi)
\end{align}

We now suppose $e_j = \bbrack{l, h - 1}$:

\begin{align}
    &\Pr(x | x \in \bbrack{l, h-1}, \mu, \pi) =\\
    &\begin{aligned}
        &\sum_{y=x + 1}^{h-1} \Pr(\bbrack{l, y - 1} | \bbrack{l, h-1}, \mu, \pi) \Pr(x | x \in \bbrack{l, y - 1}, \mu, \pi) \\
        &+ \Pr(\set{x} | \bbrack{l, h-1}, \mu, \pi) \Pr(x | x \in \set{x}, \mu, \pi) \\
        &+ \sum_{y=l}^{x - 1} \Pr(\bbrack{y + 1, h-1} | \bbrack{l, h-1}, \mu, \pi) \Pr(x | x \in \bbrack{y + 1, h-1}, \mu, \pi)
    \end{aligned} \\
    &= \begin{aligned}
        &\frac{1}{h - l} \sum_{y=x + 1}^{h-1} \left[ \pi \indickronecker{\mu < y} + (1-\pi) \frac{y - l}{h - l} \right] \Pr(x | x \in \bbrack{l, y - 1}, \mu, \pi) \\
        &+\frac{1}{h - l} \left[ \pi \indickronecker{\mu = x \lor (x = l \land \mu \leq x) \lor (x = h - 1 \land \mu \geq x)} + (1 - \pi) \frac{1}{h-l} \right] \\
        &\Pr(x | x \in \set{x}, \mu, \pi) \\
        &+\frac{1}{h - l} \sum_{y=l}^{x - 1} \left[ \pi \indickronecker{\mu > y} + (1-\pi) \frac{h - y - 1}{h - l} \right] \Pr(x | x \in \bbrack{y + 1, h-1}, \mu, \pi)
    \end{aligned}
\end{align}

As $\Pr(x | x \in \set{x}, \mu, \pi) = 1$ this allows to compute the probability of $x$ being in the interval $\bbrack{l, h-1}$ recursively.

As:
\begin{equation}
    \Pr(x | x \in \bbrack{l, y - 1}, \mu, \pi) = \Pr(x - l | x - l \in \bbrack{0, y - l - 1}, \max(0, \mu - l), \pi)
\end{equation}

We can rewrite the previous equation as:

\begin{align}
    h\Pr(x | x \in \bbrack{0, h - 1})
    &= \sum_{y = x + 1}^{h - 1} \left[ \pi \indickronecker{\mu < y} + (1-\pi) \frac{y}{h} \right] \Pr(x | x \in \bbrack{0, y - 1}, \mu, \pi) \\
    &+ \pi \indickronecker{\mu = x \lor (x = 0 \land \mu \leq x) \lor (x = h - 1 \land \mu \geq x)} + (1 - \pi) \frac{1}{h} \\
    &+ \sum_{y = 0}^{x - 1} \left[ \pi \indickronecker{\mu > y} + (1-\pi) \frac{h - y - 1}{h} \right] \Pr(x - y - 1 | x - y - 1 \in \bbrack{0, h - y - 2}, \max(0, \mu - y - 1), \pi)
\end{align}


We can now prove that $\forall x \in \bbrack{0, h - 1}, \forall \mu \in \bbrack{0, h - 1}, \pi \mapsto \Pr(x | x \in \bbrack{0, h - 1}, \mu, \pi)$ is concave on $[0, 1]$


\begin{lemma}[Log concavity affine times polynomial]
    \label{lemma:concavity_log_polynomial_times_affine}
    Let $P$ a $\log$-concave polynomial postive polynomail (for all $x$ considered) and $a, b \in \RR$ with $ax + b \geq 0$. Then $f: x \mapsto (ax + b)P(x)$ is $\log$-concave.
\end{lemma}
\begin{proof}
    Using the lemma~\ref{lemma:concavity_log_composed_functions} we have that $P'(x)^2 - P(x)P''(x) \geq 0$.
    
    As
    \[f'(x)^2 - f(x) f''(x) = a^2 P(x)^2 + (ax + b) \left[ P'(x)^2 - P(x)P''(x) \right] \] 
    we have that $f'(x)^2 - f(x) f''(x) \geq 0$ hence using the lemma~\ref{lemma:concavity_log_composed_functions} we have that $f$ is $\log$-concave.
\end{proof}


\begin{thm}[Log concavity of the BOS model]
    $\forall x \in \bbrack{0, h - 1}, \forall \mu \in \bbrack{0, h - 1}, f: \pi \mapsto \Pr(x | x \in \bbrack{0, h - 1}, \mu, \pi)$ is $\log$-concave on $[0, 1]$ and a postive (for $\pi \in [0, 1]$) polynomial of degree less than $h - 1$.
\end{thm}
\begin{proof}
    We proceed by induction on $h$:

    Initialization: $h = 1$: 
    \[ \forall x \in \bbrack{0, h - 1}, \forall \mu \in \bbrack{0, h - 1}, \Pr(x | x \in \bbrack{0, h - 1}, \mu, \pi) = 1\] which is $\log$-concave and a positive polynomial of degree $0$.

    Induction: Suppose the theorem holds for $h - 1$ and let us prove it for $h$.

    Using the previous formula we have that $f$ is a sum of psotive affine function in $\pi$ times $\Pr(x | x \in \bbrack{0, y - 1}, \mu, \pi)$ which is $\log$-concave by induction hypothesis and a positive polynomial of degree $y - 1$. We immediately deduce that $f$ is postive and polynomial of degree less than $h - 1$. Moreover using the previous lemma~\ref{lemma:concavity_log_polynomial_times_affine} we have that $f$ is $\log$-concave.

    Hence the theorem holds for $h$.
\end{proof}
