\subsubsection{Parameter estimation}

We want to estimate $\pi$ and $\mu$ from a sample $(1, \dots, m)$ with weights $W \in \RR_{+}^m$ generated by the GOD model. We aim at maximizing the likelihood of the sample : $\Pr(W | \pi, \mu)$. We proceed as explained in section~\ref{sec:univariate_generic_estimation}. All the proofs of this section are detailed in appendix~\ref{appendix:bos_proofs}, we only give here the main results.

\paragraph{Likelihood evaluation}

\begin{thm}[Likelihood is polynomial]
    $\forall m \in \NN^*, \forall x \in \bbrack{1, m}, \forall \mu \in \bbrack{1, m}$,:
    \[ \pi \mapsto \Pr(x | \mu, \pi) \]
    is a polynomial function of degree at most $m - 1$.  
\end{thm}

\begin{definition}
    We denote by $u[\mu, x, d]$ the coefficient of the polynomial of degree $d$ in the polynomial expansion of $\pi \mapsto \Pr(x | \mu, \pi)$ \textit{ie}:
    \[ \Pr(x | \mu, \pi) = \sum_{d = 0}^{m - 1} u[\mu, x, d] \pi^d \]
\end{definition}

We show that the likelihood of a single observation is a polynomial function of degree at most $m - 1$ in $\pi$. This result is important because, once the coefficients are known, it implies that the likelihood of a single observation can be computed in $\Theta(m)$ and the $\log$-likelihood of a sample (which has at most $m$ distinct values) in $\Theta(m^2)$ operations. 
For a fixed $m$, there is only $m^2$ polynomials to compute and store (one for each couple $(x, \mu)$), which leads to $m^3$ coefficients to store. We will show in the next paragraph~\ref{sec:bos_coefficients} that these coefficients can be computed in $\mathcal O(m^5)$ operations.

\paragraph{Computing the coefficients}
\label{sec:bos_coefficients}

\begin{definition}
    To simplify the comprehension of the following results, we introduce a notation for the probability in the case of $h$ categories~\footnote{This notation may seem different from the one used in the appendix but it is equivalent.}:
    \[ \bosl{x}{\mu}{h} := \Pr(x + 1 | \mu + 1, \pi) \text{ with }h\text{ categories}\]
\end{definition}

\begin{thm}[Computing the likelihood]
    \label{thm:computing_likelihood_bos}
    $\forall m \in \NN^*, \forall x \in \bbrack{1, m}, \forall \mu \in \bbrack{1, m}, \forall \pi \in [0, 1]$:

\begin{equation}
    \begin{aligned}
        \bosl{x}{\mu}{h}
        &=\frac{1}{h} \sum_{y = x + 1}^{h - 1} \bosl{x}{\mu}{y} \left[ \left( \indickronecker{\mu < y} - \frac{y}{h} \right) \pi + \frac{y}{h} \right] \\
            &+\frac{1}{h} \ \qquad \left[ \left( \indickronecker{\mu = x \lor (x = 0 \land \mu \leq x) \lor (x = h - 1 \land \mu \geq x)} - 1 \right) \pi +  \frac{1}{h} \right] \\
            &+\frac{1}{h} \sum_{y = 0}^{x - 1}\bosl{x - y}{\max(0, \mu - y)}{h - y}    \left[ \left( \indickronecker{\mu > y} - \frac{h - y - 1}{h} \right) \pi + \frac{h - y - 1}{h} \right]
    \end{aligned} \\
\end{equation}
\end{thm}

This theorem gives a recursive formula to compute the likelihood of a single observation. Using this formula, we can construct a dynamic programming algorithm to compute the coefficients of the polynomials. To do this we proceed by increasing $x$, $\mu$ and $h$. We do directly the multiplication of the polynomials.

Each term of the sum require the multiplication of a polynomial of degree at most $m - 1$ by a polynomial of degree $2$ which gives a cost of $O(m)$. The sum has at most $m$ terms, which gives a cost of $O(m^2)$. We need to apply this formula for each $(x, \mu, h)$ which gives a cost of $O(m^5)$.

\paragraph{Log concavity}

\begin{thm}[Log concavity of the BOS model]
    $\forall m \in \NN^*, \forall x \in \bbrack{1, m}, \forall \mu \in \bbrack{1, m}$:
    \[\pi \mapsto \Pr(x | x, \mu, \pi) \] 
    is $\log$-concave on $[0, 1]$.
\end{thm}

As explained in section~\ref{sec:univariate_generic_estimation}, we directly have that $\forall \mu \in \bbrack{1, m}, \pi \mapsto L_W(\pi, \mu)$ is concave. Hence we can use a ternary search algorithm to estimate $\pi$ for a given $\mu$.

\paragraph{title}

As presented in section~\ref{sec:univariate_generic_estimation}, we can estimate $\mu, \pi$ in $\mathcal O(m^3 \log \frac{1}{\epsilon})$ operations once the coefficients $u$ are computed. The coefficients $u$ can be computed in $\mathcal O(m^5)$ operations and stored in $\mathcal O(m^3)$ space. This is a major improvement compared to the EM algorithm proposed in~\cite{biernacki2016model}. Indeed we give a fully polynomial time algorithm with precision guarantees, while the EM algorithm has no guarantees on the precision and the proposed algorithm is exponential in the number of categories.

\tr{TODO: add a the run time comparisons}



