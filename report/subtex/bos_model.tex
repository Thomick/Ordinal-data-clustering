
\subsection{Stochastic Binary Ordinal Search} 

The BOS model is inspired by a standard binary search with added noise in the comparison. Consequently, the algorithm may at times misidentify the next subset for the search, ultimately causing it to overlook the sought-after value.


\subsubsection{Probabilistic model}

The stochastic binary ordinal search unfolds as follows: Let $m$ be the number of categories. Then, for at most $m-1$ steps, we perform the following three operations.
We start with the full set of categories, denoted as $e_1 = \bbrack{1,m}$. Then we perform the following steps:

At step $j$, we start with a subset of all the categories, denoted as $e_j = \bbrack{l_j, u_j - 1} \subseteq \bbrack{1,m}$.

\begin{enumerate}
    \item Sample a breakpoint $y_j$ uniformly in $e_j$ ($y_j \sim \mathcal{U}(e_j)$).
    \item Draw an accuracy indicator $z_j$ from a Bernoulli distribution with parameter $\pi$ ($z_j \sim \text{Bernoulli}(\pi)$).
    A value of $z_j=1$ indicates that the comparison is perfect, and the next step will be computed optimally. A value of $z_j=0$ implies a blind comparison at the next step.
    \item Determine the new subset $e_{j+1}$ for the next iteration. Firstly, split the subset into three intervals, namely $e_j^- = \bbrack{l_j, y_j - 1}$, $e_j^= = \set{y_j}$, and $e_j^+ = \bbrack{y_j + 1, u_j - 1}$. $e_{j+1}$ will be chosen among these intervals. If the comparison is blind ($z_j=0$), randomly select the interval with a probability proportional to its size. Alternatively, if $z_j=1$ and the comparison is perfect, select the interval containing $\mu$ (or, by default, the one closest to it).
\end{enumerate}
After $m-1$ steps, the resulting interval contains a single value, which is the observed result $e_m=\set{x}$ of the BOS model.


\begin{figure}[htbp]
\centering
\begin{adjustbox}{width=\textwidth}
\begin{tikzpicture}
  \node[obs]                               (xi) {$x_i$};
  \node[latent, left=of xi]               (em) {$e_m$};
  \node[latent, below=of em]               (zm1) {$z_{m-1}$};
  \node[latent, left=of em]               (ym1) {$y_{m-1}$};
  \node[const, left=of ym1]               (dots2) {$\ldots$};
  \node[latent, left=of dots2]               (ej) {$e_{j}$};
  \node[latent, below=of ej]               (zj) {$z_{j}$};
  \node[latent, left=of ej]               (yj) {$y_{j}$};
  \node[const, left=of yj]               (dots1) {$\ldots$};
  \node[latent, left=of dots1]               (e2) {$e_2$};
  \node[latent, below=of e2]               (z1) {$z_1$};
  \node[latent, left=of e2]               (y1) {$y_1$};
  \node[latent, left=of y1]               (e1) {$e_1$};
  \node[const, above=of ej]    (mu) {$\mu$};
  \node[const, below=of zj, yshift=-.25cm]  (pi) {$\pi$};

  \edge{e1}{y1};
  \edge{y1}{e2};
  \edge{z1}{e2};
  \edge{e2}{dots1};
  \edge{dots1}{yj};
  \edge{yj}{ej};
  \edge{zj}{ej};
  \edge{ej}{dots2};
  \edge{dots2}{ym1};
  \edge{ym1}{em};
  \edge{zm1}{em};
  \edge{em}{xi};
  \edge {mu} {e2, ej, em};
  \edge {pi} {z1,zj, zm1};

  \plate[inner sep=.25cm] {} {(xi)(e1)(zm1)(z1)} {$i=1, \ldots, n$};
\end{tikzpicture}
\end{adjustbox}
\caption{Graphical model of the stochastic Binary Ordinal Search.}
\label{fig:graphical_model}
\end{figure}

%\ar{Do a figure for the Mixture case}
%\tm{Maybe, although it does not give much more information since it can be trivially derived from this one. Good job for the figure by the way}
%\ar{Agreed, and probably too complicated to do anyways}


