\section{Conclusion}
In this study, we analyzed model-based clustering for ordinal data, with a specific focus on the Binary Ordinal Search (BOS) and a proposed alternative we called Globally Ordinal Distribution (GOD) models. We aimed to understand and evaluate their efficiency in clustering and classifying ordinal data compared to more traditional methods like K-Means and Gaussian Mixture Models. Our exploration spanned both synthetic and real-world datasets, providing a comprehensive view of the models' performance in various scenarios.

The experiments on synthetic data confirmed the theoretical foundations of the BOS and GOD models. When parameters were known, both models showed an ability to recover the underlying structure of the generated data. Particularly, the BOS model, despite its computational intensity due to its design, performed well in clustering tasks, highlighting its potential for applications with ordinal data. The GOD model, with its more manageable computational requirements, also demonstrated promising results, making it a practical alternative for larger datasets.

When applied to real-world datasets, the results were more nuanced. While both BOS and GOD models performed competitively in certain scenarios, they did not universally outperform the traditional methods. This suggests that while specialized ordinal models are interesting, especially in scenarios where the ordinal nature of data is pronounced, they are not a default solution. It is essential to consider the specific characteristics of the dataset and the computational resources available when choosing the appropriate clustering method.

Different visualizations also provide further insights into how the models partition the data space. They revealed that while the clusters identified by the BOS and GOD models often made intuitive sense, they sometimes differ significantly from those identified by K-Means and Gaussian Mixture Models. This highlights the different assumptions and approaches these models take when learning the structure within data.

In conclusion, the study reaffirms the potential of model-based clustering for ordinal data, particularly highlighting the BOS and GOD models as valuable tools. However, it also demonstrates that the choice of model should be informed by both the nature of the data and the practical constraints of the problem at hand. Further research could explore further refinements to these models, more extensive comparisons with other methods, and applications to a broader range of real-world scenarios.

\section{Contribution statement}
This project reflects a collaborative effort where all three students, Ali \textsc{Ramlaoui}, Thomas \textsc{Michel} and Théo \textsc{Rudkiewicz}, made equal and substantial contributions. 
We give here the main but not exclusive focus of each one:
\begin{itemize}
    \item Ali \textsc{Ramlaoui} focused on the experiments and the implementation of the AECM algorithm.
    \item Thomas \textsc{Michel} implemented the BOS model and the EM algorithm for the BOS model.
    \item Théo \textsc{Rudkiewicz} developed the GOD model and the algorithm associated with it.
\end{itemize}