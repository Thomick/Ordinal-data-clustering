\section{Conclusion}
In this study, we analyzed model-based clustering for ordinal data, with a specific focus on the Binary Ordinal Search (BOS) and a proposed alternative we called Globally Ordinal Distribution (GOD) models.
We provide an efficient algorithm for estimating the parameters of these models. This include a polynomial algorithm for the BOS model, improving the exponential algorithm proposed in the literature.
We aimed to understand and evaluate their efficiency in clustering and classifying ordinal data compared to more traditional methods like K-Means and Gaussian Mixture Models. Our exploration spanned both synthetic and real-world datasets, providing a comprehensive view of the models' performance in various scenarios.

The experiments on synthetic data confirmed the theoretical foundations of the BOS and GOD models. 
When applied to real-world datasets, the results were more nuanced. While both BOS and GOD models performed competitively in certain scenarios, they did not universally outperform the traditional methods. This suggests that while specialized ordinal models are interesting, especially in scenarios where the ordinal nature of data is pronounced, they are not a default solution. It is essential to consider the specific characteristics of the dataset and the computational resources available when choosing the appropriate clustering method.

Different visualizations also provide further insights into how the models partition the data space. They revealed that while the clusters identified by the BOS and GOD models often made intuitive sense, they sometimes differ significantly from those identified by K-Means and Gaussian Mixture Models. This highlights the different assumptions and approaches these models take when learning the structure within data.

In conclusion, the study reaffirms the potential of model-based clustering for ordinal data, particularly highlighting the BOS and GOD models as valuable tools. However, it also demonstrates that the choice of model should be informed by both the nature of the data and the practical constraints of the problem at hand. Further research could explore further refinements to these models, more extensive comparisons with other methods, and applications to a broader range of real-world scenarios. Moreover, it might also be interesting to look for an even more efficient way to compute the coefficients for the likelihood of the GOD model.
