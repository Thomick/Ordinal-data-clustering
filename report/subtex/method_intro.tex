\section{Method}

Our goal is to cluster multivariate data. Each dimension contains data generated by a random process characterized by parameters dependent on the cluster. To estimate the cluster, one can utilize the AECM algorithm, as introduced in~\citep{meng1997algorithm}. This algorithm is quite versatile, requiring only the ability to estimate the univariate parameters of a weighted set of data points. We present this algorithm in section~\ref{sec:aecm}.

Similar to the approach in~\cite{biernacki2016model}, we concentrate on ordinal data. Therefore, we assume that each dimension follows a process that generates ordinal categories. In section~\ref{sec:univariate}, we introduce two random processes for modeling ordinal data: the binary ordinal search (BOS) model from~\cite{biernacki2016model} and the globally ordered data (GOD) model.

To apply the proposed methods, each dimension of the data should represent an ordinal category. These categories must adhere to the following properties:

\begin{itemize}
\item The categories must be well-ordered (ordinal data): They are linearly ordered, and each non-empty subset contains a least element. This implies that any element can be compared to any other, and we can enumerate all the categories in increasing order.

\item The set of categories is finite. This simplifies the previous assumption to the existence of a linear ordering. This assumption is necessary to ensure that the stochastic search terminates after a fixed number of steps, implying a finite number of possible runs of the search.

\end{itemize}



