\section{Method}
The approach taken in this report follows the one proposed by \cite{biernacki2016model}. It involves defining a probabilistic model based on the observation of categorical data. More precisely, the authors introduce a Stochastic Binary Search process that leads to the selection of a category. 

For this particular model, we need to make the assumption that:
\begin{itemize}
    \item The categories must be well-ordered (ordinal data): The categories are linearly ordered, and each non-empty subset contains the least element. This implies that any element can be compared to any other, and we can enumerate all the categories in increasing order.
    \item The set of categories is finite. This simplifies the previous assumption to the existence of a linear ordering. This assumption is necessary to ensure that the stochastic search terminates after a fixed number of steps, implying a finite number of possible runs of the search.
\end{itemize}
We will first define the binary ordinal search (BOS) model in the univariate case. The multivariate variant can be deduced by applying a similar search to each of the features independently. Then, we will discuss how to estimate the parameters of the model from a set of samples following the general method of the Expectation-Maximization algorithm (EM). Finally, we will show how the original authors extend EM to cluster ordinal data.

\subsection{Probabilistic model}


