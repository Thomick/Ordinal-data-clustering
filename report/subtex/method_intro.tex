\section{Method}

We suppose that we aim to cluster multivariate data. Each dimension has data generated by a random process parametrized by parameters that are dependant of the cluster. To estimate the cluster it is possible to use the AECM algorithm introduced in~\citep{meng1997algorithm}. This algorithm is quite generic and only requires to be able to estimate the univariate parameters of a weighted set of data points. We present this algorithm in section~\ref{sec:aecm}.

Similarly to~\cite{biernacki2016model} we focus on ordinal data. Therefore we suppose that each dimension follow a process that generates ordinal categories. 
In section~\ref{sec:univariate}, we present two random processes to model ordinal data the binary ordinal search (BOS) model from~\cite{biernacki2016model} and the globally ordered data (GOD) model.

To apply the proposed methods the each dimension of the data should represent an ordinal categorie. These categories must respect the following properties:

\begin{itemize}
\item The categories must be well-ordered (ordinal data): The categories are linearly ordered, and each non-empty subset contains the least element. This implies that any element can be compared to any other, and we can enumerate all the categories in increasing order.

\item The set of categories is finite. This simplifies the previous assumption to the existence of a linear ordering. This assumption is necessary to ensure that the stochastic search terminates after a fixed number of steps, implying a finite number of possible runs of the search.
\end{itemize}



